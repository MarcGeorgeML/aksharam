\documentclass{article}
\oddsidemargin 0.1in
\addtolength{\oddsidemargin}{-.7in}
\addtolength{\textwidth}{2.75in}
\addtolength{\textheight}{2in}
\addtolength{\headheight}{-1in}
\pagenumbering{gobble}


\title{Muthoot Institute of Technology \& Science, Varikoli
\\Department of Artificial Intelligence \& Data Science
\\B. Tech. Computer Science(AI)
\\Semester 6 
\\CAD 334 : Mini Project
\\ \textbf{ABSTRACT}}

\author{
    Marc George : MUT22CA043\\
    Merlin Sarah Jiju : MUT22CA045\\
    Timon K John : MUT22CA068
}

\begin{document} 

\maketitle
	

\begin {center} {\large {\bf Education Platform for Malayalam Language}} \end {center}
{\textit{\textbf{Keywords: Handwritten Character Recognition, Malayalam Script Learning, AI-Powered Learning, Language Acquisition Technology, Interactive Learning Application, Regional Language Education, Handwriting Feedback System, Deep Learning in Education}}}

\section{Abstract}
\hspace{1cm}\textbf{Problem Statement:}
The increasing popularity of language learning applications presents a significant challenge for beginners, particularly those learning regional scripts like Malayalam. One of the primary issues users face is the difficulty in mastering the writing of characters, especially without adequate feedback on their handwriting. Many learners struggle to develop accurate handwriting without personalized guidance, which often leads to setbacks and slow progress. Additionally, conventional methods for learning writing may not offer real-time feedback, hindering the development of proper writing techniques. While schools may provide formal education on writing the language, this application is primarily designed for self-learning, offering users the flexibility to practice and improve their handwriting independently.\\

\noindent\hspace{1cm}\textbf{Proposed Solution:} 
This work proposes a solution that combines interactive learning with handwritten character recognition to address these challenges. The proposed system allows users to write Malayalam letters, capture their handwriting, and receive real-time feedback on their accuracy through an AI-powered recognition model. The system incorporates a progressive learning approach, dividing the language script into levels that gradually increase in difficulty. By utilizing a machine learning model trained on a diverse set of handwritten Malayalam characters, the app ensures high accuracy in recognizing user inputs. The application features a user-friendly interface that encourages consistent practice and provides personalized feedback, making it suitable for beginners and children. The effectiveness of the proposed system is evaluated through user interactions and the accuracy of handwriting recognition over time.


\section{Technologies using}
\begin{itemize}
    \item PyTorch
    \item OpenCV
    \item Django
    \item React JS
\end{itemize}

\bibliographystyle{unsrt}	

\begin{thebibliography}{999}

\bibitem{1} V.K. Vaishakh, B.D. Lyla \textquotedblleft Handwritten Malayalam Character Recognition System using Artificial Neural Networks\textquotedblright, \textit{2020}.  

\bibitem{2} K.B. Baiju, T.S. Sabna, V.L. Lajish \textquotedblleft Segmentation of Malayalam Handwritten Characters into Pattern Primitives and Recognition using SVM\textquotedblright, \textit{2020}.  

\bibitem{3} K. Manjusha, M. Anand Kumar, K.P. Soman \textquotedblleft On developing handwritten character image database for Malayalam language script\textquotedblright, \textit{Engineering Science and Technology, an International Journal}, 2019.  

\bibitem{4} S. Anish, V. Preeja \textquotedblleft A Novel Method for Malayalam Handwritten Character Recognition\textquotedblright, \textit{2015}.  

\bibitem{5} M. Abdul Rahiman, M.S. Rajasree \textquotedblleft An Efficient Character Recognition System for Handwritten Malayalam Characters Based on Intensity Variations\textquotedblright, \textit{2011}.  
.



\end{thebibliography}
\vspace{0.2in}
\hrule
\vspace{.3in}

\hspace{-.25in}
\begin{tabbing}
xxxxxxxxxxxxxxxxxxxxxxxxxxxxxxxxxxxxxxxxxxxxxxx\= xxxxxxxxxxxxxxxxxx\= \kill

Project Coordinators \>\>\>\> Project Guide
\\
\\
\\
Ms. Rinu Rose George \& Mr. Shyam Krishna K, \>\> Ms.  Philo Sumi \\
Assistant Professor \>\> Assistant Professor\\
Dept. of AI \& DS \>\> Dept. of AI \& DS\\
MITS \>\> MITS
\end{tabbing}

\end{document}